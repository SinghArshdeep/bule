\documentclass{ecai}
\usepackage{times}
\usepackage{graphicx}
\usepackage{latexsym}
\usepackage[utf8]{inputenc}
\usepackage{graphicx}
\usepackage{xspace}
\usepackage{amsmath}
\usepackage{booktabs,multirow}
\usepackage{verbatim}
\usepackage{graphics}
\usepackage{hyperref}
\newcommand{\citet}{\cite}

\newcommand{\TODO}[1]{\textcolor[rgb]{1.00,0.00,0.00}{todo: #1} }
\newcommand{\VALE}[1]{\textcolor[rgb]{0.00,0.00,0.54296875}{vale: #1} }


\newcommand\bigforall{\mbox{\Large $\mathsurround0pt\forall$}}
\newcommand\bigexists{\mbox{\Large $\mathsurround0pt\exists$}}

\usepackage{acronym}
\acrodef{GTTT}{Generalized Tic-tac-toe}
\acrodef{MCTS}{Monte Carlo Tree Search}
\acrodef{PNS}{Proof Number Search}
\acrodef{QBF}{Quantified Boolean Formula}
%\acrodefplural{QBF}[QBF]{Quantified Boolean Formulae}
\acrodefplural{QBF}{Quantified Boolean Formulae}

\newcommand{\white}{\textsc{W} }
\newcommand{\black}{\textsc{B} }
\newcommand{\board}{{\mathsf{board}}}
\newcommand{\timee}{\mathsf{time}}
\newcommand{\lose}{\mathsf{lose}}
\newcommand{\win}{\mathsf{win}}
\newcommand{\move}{\mathsf{move}}
\newcommand{\occupied}{\mathsf{occupied}}
\newcommand{\ld}{\mathsf{ladder}}
\newcommand{\cnt}{\mathsf{count}}
\newcommand{\moveL}{\mathsf{moveL}}
\newcommand{\depth}{\mathsf{F}}
\newcommand{\size}{\mathsf{N}}
\newcommand{\ENC}{\emph{COR}\xspace}

\begin{document}

\title{BULE: A Simple Grounding Language for SAT}
\author{Valentin Mayer-Eichberger\institute{Technische Universit\"at Berlin, Germany}}

\maketitle

\begin{abstract}
    Bla bla bule bule gringo gringo
\end{abstract}

\section{Introduction}
\section{Definition}
\section{Conclusion}

This is the counter encoding found in \cite{Sinz2005}.

We thank Cameron Browne, Ryan Hayward, and Bjarne Toft for providing us with
the hand-crafted Hex puzzles. We also thank Mikoláš Janota for his feedback on
earlier versions of the encoding.

\bibliographystyle{ecai}
\bibliography{main}

\end{document}
