\documentclass[runningheads]{llncs}
\usepackage{misc}
\usepackage{cite}
\usepackage{amsmath,amssymb,amsfonts}
\usepackage{algorithmic}
\usepackage{graphicx}
\usepackage{textcomp}
\usepackage{xcolor}
\usepackage{xspace}
\usepackage{color}
\usepackage{multirow}
\usepackage{textcomp}
\usepackage{booktabs}
\newcommand{\nn}{\raisebox{0.5ex}{\texttildelow}}
\definecolor{codegreen}{rgb}{0,0.6,0}
\definecolor{codegray}{rgb}{0.5,0.5,0.5}
\definecolor{codepurple}{rgb}{0.58,0,0.82}
\definecolor{backcolour}{rgb}{0.95,0.95,0.92}
\usepackage{listings}
\lstdefinestyle{mystyle}{
    backgroundcolor=\color{backcolour},   
    commentstyle=\color{codegreen},
    keywordstyle=\color{magenta},
    numberstyle=\tiny\color{codegray},
    stringstyle=\color{codepurple},
    basicstyle=\ttfamily\footnotesize,
    breakatwhitespace=false,         
    breaklines=true,                 
    captionpos=b,                    
    keepspaces=true,                 
    numbers=left,                    
    numbersep=5pt,                  
    showspaces=false,                
    showstringspaces=false,
    showtabs=false,                  
    tabsize=2,
    literate={~}{\nn}{1}
    %literate={~}{\raise.17ex\hbox{$\scriptstyle\sim$}}{1},
    %literate={~} {$\sim$}{1}
%    literate={~}{\textasciitilde}{1},
%    literate={~}{$\neg$}{1},
}
\lstset{style=mystyle}

\newcommand\bigforall{\mbox{\Large $\mathsurround0pt\forall$}}
\newcommand\bigexists{\mbox{\Large $\mathsurround0pt\exists$}}

\newtheorem{example}{Example}

\newcommand{\complexityclass}[1]{\textsf{\footnotesize #1}\xspace}
\newcommand{\ptime}{\complexityclass{P}}
\newcommand{\np}{\complexityclass{NP}}
\newcommand{\pspace}{\complexityclass{PSPACE}}
%\newcommand{\exptime}{\complexityclass{EXPTIME}}
%\newcommand{\nexptime}{\complexityclass{NEXPTIME}}
%\newcommand{\nexpspace}{\complexityclass{EXPSPACE}}



%% Call on 2021-03-13
%% 
%% 
%% Motivation: 
%% 
%% simpler and more strict, yet expressive enough. So it's more well-behaved and defined. 
%% This could lead solver developers to go for the actual non-ground specification of the problem. 
%% 
%% * flat
%% * variables,but no arithmetic 
%% * everything
%% 
%% Orthogonal to this hierarchy is the hierarchy of quantifier levels. 
%% And a problem can be horn (max 1 positive literal per clause) or not. 
%% 
%% focus on grounding language (not fast grounder yet)
%% 
%% simpler and more strict: 
%% * non-recursive 
%% * separation of grounding and search variables
%%     * Because we make this distinction, we can rule out recursion. 
%% * no functional terms
%%  
%% Comment to Abdallahs Ideas
%% 
%% #A[T], a[X], b[Y] :: a(X)?
%% #E[T], a[X], b[Y] :: b(X)?
%% 
%% a[X], b[Y] :: c(X,Y) | d(X). 
%% 
%% a[X], b[Y], c[R] : k[R] :: b[Y] : c[X] : c(X,Y) | e[X] : d(X). 
%% 
%% a[X], b[Y] :: b[Y], c[X] : c(X,Y) | e[X] : d(X). 
%%  
%% Pseudo-Boolean: 
%% a[X], b[Y] :: 12 { b[Y], c[X] : X^2 * c(X,Y) } 12 | e[X] : d(X). 
%% 
%% Call on 2021-03-13
%% 
%% Contributions
%% - SAT Programming
%% actually, a QSAT Programming Language: PSpace 
%% We fix the hebrand terms via (LOOK AT THIS: some problems cant be expressed nicely in BULE?)
%% There is a working implementation, used in practice. 
%% Formal analysis of non-ground modelling language 
%% Much cleaner than ASP for NON-Ground modelling lanugage 
%% ASP(ground) is sigma2P, 
%% ASP is not intended for modelling PSPACE languages
%% Loop formulas> https://dl.acm.org/doi/abs/10.1145/1131313.1131316 SAT vs ASP
%% 1. Fix Format 
%% 2. fix the Examples for a MVP(minimal viable paper).
%% Why iterators in nice and not core? (can be expressed with extra variables)
%% Debate: Should we allow the following: 
%% Exponents? facts[2..2^K].
%% Free variables?
%% Syntax 


\title{ QBF Programming with \emph{Bule}}

\author{Jean Christoph Jung\inst{1} \and Valentin Mayer-Eichberger\inst{2} \and
Abdallah Saffidine\inst{3}}
\authorrunning{Jung et.al.}

\institute{Universit\"at Bremen, Germany \and Technische Universit\"at Berlin, Germany \and University of New South Wales, Sydney, Australia }

\begin{document}

\maketitle

\begin{abstract}
    Bule introduces the concept of QBF Programming - a paradigm that complements QBF solving with an intuitive modelling language. 
    The language is easy to learn and has a transparent grounding process that rolls out term variables to a CNF formula with a quantifier prefix. 
%    Bule separates domain predicates for the grounding process and search variables syntactically. 
    The semantics is strict and well-behaved such that we can show complexity classes for both ground and non-ground programs. 
%    Our implementation can connect to any QBF solver and in case of only existential variables SAT solver. 
    Finally, Bule is a tool to learn QBF and SAT Programming due to its easy format and smooth integration with any QBF and SAT solvers. 
\end{abstract}

\section{Introduction and Motivation}

\paragraph{From SAT to QBF.}

\paragraph{Running Example}

\paragraph{Study on encodings for QBF rather new.}

\paragraph{QBF-thinking as a paradigm.} Difference between ASP and SAT, Modelling Language. Thinking in non-ground and compilation language. 

\section{The Bule Language}

Brief Syntax and Semantics

\subsection{Complexity}

\paragraph{Fragments that lead to interesting classes.}

\paragraph{Table}

\abd{Analyse} and \jean{check}. 

In Table \ref{tab:complexity} we give an overview of the complexity classes
that each fragment of is complete for. The proof follows from the reduction of
Turing machines into an expression in Bule using subset of constructs.

\begin{table}
  \caption{Complexity Results}
  \centering
  \label{tab:complexity}
  \begin{tabular}{lccc}
    \toprule
    Clauses  & \multicolumn{1}{c}{Horn}  & \multicolumn{2}{c}{Any} \\
    \cmidrule{3-4}
    $\forall$ Quantifiers   & $\cmark$ & $\xmark$ & $\cmark$ \\
    \midrule
    \bflat   & \ptime    & \np       & \pspace  \\
    \bcore   & \exptime & \nexptime  & \expspace  \\
    \bfull   & &     &   \\
    \bottomrule
  \end{tabular}
\end{table}


\section{Examples}

\paragraph{Counter}
\paragraph{SAT Reachability}
\paragraph{QBF Reachability}
\paragraph{Turing Machine}

\section{Related Work}
 \begin{itemize}
     \item Writing Declarative Specification for Clauses: \cite{Gebser16} 
         \begin{itemize}
             \item The proposed language is a syntactic alternative to \bcore
             \item It uses translation forth and back between ASP
             \item no complexity for non-ground language analysed. 
             \item Does not distinguish syntactically between grounding facts and search variables as we do. 
             \item \bnice provides convenient modelling support that their language does not have. Implicit variables for instances. 
         \end{itemize}
     \item ASP vs SAT: Why are there so many Loop Formulas. (reachability to SAT without additional variables is large): \cite{Lifschitz04} .
     \item Programming a SAT formula via frameworks (e.g. \cite{Pysat18},
         more?), then this is a imperative description, but NOT a declarative
         one! Complexity results on Python are undecided of course. 
     \item Disjunctive Datalog, DLV \cite{Eiter97} or even just Datalog \cite{Gottlob89}
     \item Relate it to Complexity of different flavors of logic programming \cite{Gottlob01}
     \item Assat, translating ASP programs to SAT \cite{Lin04}
     \item \cite{Janhunen11}. Compact Translations of Non-disjunctive Answer Set
         Programs to Propositional Clauses
     \item FO(ID). First order logic with bounds (guards) \cite{Wittocx10}
     \item Predicate Logic as a modeling language. IDP System. \cite{Cat18}
     \item Answer Set Programming Lparse/Gringo \cite{Gebser15}, \cite{Ferraris05}
 %    \item QBF Solvers \cite{Lonsing17,Tentrup15}
     \item Lazy Clause Generation Interleaving Grounding and Search \cite{Cat15}
     \item Relationship with Effectively Propositional Logic (EPL) \cite{Moura08}? 
     \item http://picat-lang.org/
 \end{itemize}

\section{Conclusion and Further Work}

\bibliographystyle{plain}
\bibliography{main}

\end{document}


%%%%%%%%%%%%%%%%%%%%%%%%%%%%%%%%%%%%%%%%%%%%%%%%%%%%%%%%%%%%%%%%%%%%%%%%%%%
%%%% Previous state 15.03.2021 Before Rewrite for SAT 
%%%%%%%%%%%%%%%%%%%%%%%%%%%%%%%%%%%%%%%%%%%%%%%%%%%%%%%%%%%%%%%%%%%%%%%%%%%
%% 
%% 
%% \title{ SAT Programming with \emph{Bule}}
%% 
%% \author{Jean Christoph Jung\inst{1} \and Valentin Mayer-Eichberger\inst{2} \and
%% Abdallah Saffidine\inst{3}}
%% \authorrunning{Jung Mayer-Eichberger Saffidine}
%% 
%% \institute{Universit\"at Bremen, Germany \and Technische Universit\"at Berlin, Germany \and University of New South Wales, Sydney, Australia }
%% 
%% \begin{document}
%% 
%% \maketitle
%% 
%% \begin{abstract}
%%     SAT Programming extends SAT Solving by a declarative modelling language. 
%%     We propose Bule, a Logic Programming language, based on classical logic that introduces SAT Programming. 
%%     A ground program in Bule consist of a set of propositional clauses with meaningful literals. 
%%     The process of translating a non-ground program to propositional clauses is called grounding and is done by iterating over the domain of their variables. 
%%     The semantics of Bule is simpler than Answer Set Programming or First Order Logic with bounds.  
%%     Bule also offers techniques based on SAT Solving such as Approximate Model Counting and Quantified Boolean Formulae 
%%     and enriches such SAT related technologies by a uniform modelling language. 
%% \end{abstract}
%% 
%% \section{Introduction and Motivation}
%% 
%% Comparing Answer set programming and satisfiability solving. Key difference
%% according to \cite{Lierler17} is that ASP has a modelling language and a
%% grounder attached to it, whereas SAT is focused on problems stemming from
%% already ground problems. We will try to change that !
%% 
%% \section{Bule Syntax and Semantics}
%% 
%% \vale{has to be updated to the hierarchy}
%% For didactic reasons, we define syntax and semantics of \bule in three stages. 
%% We start with a fully ground program without term variables in \bflat, which is essentially CNF with readable literals. 
%% The non-ground language \bcore contains the basic functionality of the modelling language. 
%% Third we present the complete language \bfull, that allows for several extensions to ease modeling. 
%% 
%% \begin{itemize}
%%     \item \bflat: 
%%         \begin{itemize}
%%             \item Set of ground clauses with meaningful identifiers 
%%         \end{itemize}
%%     \item \bcore Non-ground language with clean semantics
%%         \begin{itemize}
%%             \item Syntax: Ground facts. Term variables (Capital letter) in generators that bind clauses. $::$
%%             \item fully bound (i.e. all variables are bound by a positive generator) 
%%             \item Exponential compression in the arity of the variables. 
%% 
%%                 p[X], p[Y] ::  q(X,Y). 
%%                 q(X,Y) : p[X] : p[Y].
%% 
%%         \end{itemize}
%%     \item \bfull 
%%         \begin{itemize}
%%             \item Grounding fact definitions. (Restriction: Induced graph is acyclic)
%%             \item All term variables are bound by a generator. 
%%             \item Negated generators allow if bound positively by other generator. 
%%             \item arithmetic (fully ground), exponential compression!
%%             \item Ranges, exponential compression!
%%         \end{itemize}
%%     \item \bnice is just for modelling and close to the implementation. 
%%         \begin{itemize}
%%             \item Implicitly bound variables, i.e. variables can be free. 
%%             \item Iterators (see the encoding for ATLEASTONE)
%%         \end{itemize}
%% \end{itemize}
%% 
%% \subsection{\bflat}
%% 
%% A ground Bule program is a set of clauses with literals. 
%% 
%% \subsection{\bcore}
%% 
%% Let $\Sigma,\Omega$ be disjoint countably infinite sets of relation symbols.  
%% Each relation symbol $R\in \Sigma\cup\Omega$ has an associated arity $\mn{ar}(R)$.  
%% Let us further fix a countably infinite supply of constant symbols \mn{Con} and variable symbols $\mn{Var}$.  
%% An \emph{atom} is of the form $R(t_1,\ldots,t_n)$ for some $R\in \Sigma\cup\Omega$ and $t_1,\ldots,t_n\in\mn{Con}\cup\mn{Var}$ and $n=\mn{ar}(R)$. 
%% An atom $R(t_1,\ldots,t_n)$ is \emph{ground} if $t_1,\ldots,t_n\in\mn{Con}$; ground atoms are also called \emph{facts}. 
%% An atom $R(t_1,\ldots,t_n)$ is a \emph{$\Sigma$-atom} or an \emph{$\Omega$-atom} if $R\in \Sigma$ or $R\in \Omega$, respectively. 
%% A \emph{literal} is an atom $R(t_1,\ldots,t_n)$ or a negated atom $\neg R(t_1,\ldots,t_n)$. 
%% An \emph{instance} is a finite set of facts. A \emph{substitution} is a map $v:\mn{Var}\cup\mn{Con}\to\mn{Con}$ which is the identity on $\mn{Con}$, that is, $v(a)=a$, for all $a\in\mn{Con}$.
%% et $\Imc$ be an instance, $\alpha$ a Boolean formula over atoms, and $v$ a substitution. 
%% 
%% We define $\Dmc,v\models \alpha$ inductively as follows: 
%% %
%% \begin{align*}
%%   %
%%   \Imc,v & \models R(t_1,\ldots,t_n) && \text{if
%%   $R(v(t_1),\ldots,v(t_n))\in \Imc$} \\
%%   %
%%   \Imc, v& \models \neg \alpha && \text{if $\Imc,v\not\models \alpha$}
%%   \\
%%   %
%%   \Imc, v& \models \alpha \wedge \alpha' && \text{if $\Imc,v\models
%%   \alpha$ and $\Imc,v\models\alpha'$}
%%   %
%% \end{align*}
%% %
%% If $\Imc,v\models\alpha$, we call $v$ a \emph{model} of $\alpha$ over the instance \Imc. 
%% A \emph{clause} is a disjunction of literals.
%% 
%% A \emph{\bcore-program} is a pair $\Pi=(\Imc,\Pmc)$ where $\Imc$ is a finite set of facts and \Pmc is a set of rules of the form 
%% %
%% \[B_1, \ldots, B_k :: L_1, \ldots, L_m\]
%% %
%% where $B_1,\ldots,B_k$ are $\Sigma$-atoms and $L_1,\ldots,L_m$ are $\Omega$-literals such that every variable that occurs in one of the $L_i$ does occur in one of the $B_i$.
%% 
%% The semantics of \bcore-programs $\Pi=(\Imc,\Pmc)$ is defined via groundings. 
%% More precisely, with every \bcore-program $\Pi=(\Imc,\Pmc)$, we associate a set $\mn{cl}(\Pi)$ of ground clauses as follows.  
%% For every rule $B_1,\ldots,B_k :: L_1,\ldots, L_n\in\Pmc$ and every model $v$ of $B_1\wedge\ldots\wedge B_k$ in \Imc, $\mn{cl}(\Pi)$ contains the clause
%% %
%% \[v(L_1)\vee\ldots\vee v(L_n).\]
%% %
%% We say that $\Pi$ is \emph{satisfiable} if the set of clauses $\mn{cl}(\Pi)$ is satisfiable, that is, 
%% there is an instance $\Mmc$ such that $\Mmc,\emptyset \models\mn{cl}(\Pi)$ (note that $\mn{cl}(\Pi)$ is ground, so the empty substitution $\emptyset$ suffices).
%% 
%% {\color{red} Although \bcore is a relatively simple language, we can
%% already model natural problems with it. To distinguish $\Sigma$- and
%% $\Omega$-atoms we write them with parenthesis $p(\;\;\cdot)$ or brackets
%% $q[\;\;\cdot]$,
%% respectively.
%% 
%% 
%% %\begin{example}
%% %  
%% %\end{example}
%% 
%% }
%% 
%% 
%% \subsection{\bfull}
%% 
%% We extend \bcore with \emph{extended instances}, \emph{iterators},
%% \emph{integers}, and \emph{implicit declarations} and define the
%% semantics by mapping to \bcore. We start with definitions. 
%% 
%% An \emph{extended instance \Jmc} is a union of an instance \Imc with a
%% a set of rules of the form
%% %
%% \[B_1,\ldots,B_k\rightarrow B\]
%% %
%% where $B,B_1,\ldots,B_k$ are $\Sigma$-atoms and every variable that
%% occurs in $B$ occurs in one of the $B_i$. We further require that the
%% rules in \Jmc are acyclic in the sense that the graph $G_\Jmc = (V_\Jmc,E_\Jmc)$ is
%% acyclic, where $V_\Jmc$ is the set of all relation symbols that occur in
%% \Jmc and $(R,R')\in E_\Jmc$ if there is a rule
%% $B_1,\ldots,B_k\rightarrow B$ where $B$ is an atom with relation
%% symbol $R'$ and some $B_i$ is an atom with relation symbol $R(\cdot)$.  
%% 
%% An extended \bcore-program is now a pair $\Pi=(\Jmc,\Pmc)$ where
%% $\Jmc$ is an extended instance and \Pmc a set of rules as above. Every
%% extended \bcore-program \emph{induces} a \bcore-program $(\Imc,\Pmc)$
%% via an inductive process as follows. Let $\Imc_0$ be the set of facts
%% in \Jmc, and define $\Imc_{i+1}$ from $\Imc_i$ by adding, for every
%% rule $B_1,\ldots,B_k\rightarrow B\in \Jmc$ and every model $v$ of
%% $B_1\wedge \ldots \wedge B_k$ over $\Imc_i$, the fact $v(B)$ to
%% $\Imc_{i+1}$. Note that this process terminates due to acyclicity of
%% \Jmc; $\Imc$ is defined to be fixed point of the sequence $\Imc_n$. An
%% extended \bcore-program is satisfiable if the induced \bcore-program
%% is satisfiable. 
%% 
%% Extended \bcore-programs can be used to define auxiliary facts which
%% are often useful for modeling, e.g.~see the following example. 
%% %
%% \begin{example}
%%   %
%%   \textcolor{red}{TODO.}
%%   %
%% \end{example}
%% 
%% 
%% \subsection{\bnice}
%% 
%% To make it easier to define the semantics of \bfull we omitted the syntactic sugar. 
%% The language \bnice on the other hand contains all syntactic sugar and makes modelling very convenient.  
%% Features like: 
%% 
%% \begin{itemize}
%%   \item iterators of the form $c(I):dom[I]$
%%   \item ranges $dom[1..k]$ 
%%   \item implicit definition of term variables as in the clause $~c(I,J),b(I,J).$.
%%   \item simple arithmetic as in the rule $c(I,J),c(I+1,J).$ 
%%   \item more arithmetic as in the rule $ I== 0 \ldots Y^2, k[K] :: c(I^K\%2).$
%% \end{itemize}
%% 
%% \section{Implementation Details}
%% 
%% \vale{Program transformations. Multiple stages of grounding. Conceptual algorithm with some states.}
%% 
%% \section{Application}
%% 
%% \subsection{$4x4$ NQueens as \bflat}
%% 
%% Flat Bule only contains grounded clauses. It's essentially a pretty print of CNF.
%% 
%% \lstinputlisting{programs/queens_flat.bul}
%% 
%% \subsection{Simplified $4x4$ NQueens as \bcore}
%% 
%% Only vertical and horizontal constraints.
%% 
%% \lstinputlisting{programs/queens_core.bul}
%% 
%% \subsection{Generic NQueens in \bfull}
%% 
%% \lstinputlisting{programs/queens_full.bul}
%% 
%% Completing a setup of non-attackig queens on a chess board is infact NP-Complete (proven in 2017 \cite{Gent17}).
%% 
%% \subsection{Reachability}
%% 
%% Given a set of edges, the following program encodes reachability in the induced graph, i.e. 
%% 
%% \lstinputlisting{programs/reachability/reachability.bul}
%% 
%% \subsection{s-t Connectivity via Savitch}
%% 
%% \subsection{Cardinality Encoding}
%% 
%% Bule does not provide native cardinality constraints. 
%% There is plenty of literature on how to efficiently encode such constraints to CNF maintaining good properties. 
%% We show the \emph{counter} encoding based on \cite{Sinz05} in \bnice syntax.
%% Assume that the variables $element(1\ldots n)$ are already defined, as well as 
%% the auxiliary variables $count(I,J)$. 
%% The encoding consists of 4 different types of clauses: 
%% 
%% \lstinputlisting{programs/counter.bul}
%% 
%% Variations of this encoding can express other types of cardinality.
%% If the second last unit is omitted, the clause set defines an AtMost constraint ( respectively the last unit an AtLeast constraint). 
%% For $k=1$ the counter encoding comes down to the ladder encoding (cite Ian Gent). 
%% 
%% \subsection{Planning Problem}
%% 
%% Find a good planning problems. 
%% Sokoban might be too complicated. 
%% Is there a simpler one ? 
%% 
%% \subsection{Prime Game: QBF}
%% 
%% Here we show that Bule can be used to model PSPACE complete problems in QBF. 
%% We also show how to do LOG encoding.  
%% 
%% Two players choosing the bits of a 5 bit number from lowest to highest bit such that 
%% the second player wins if the number is prime and the first if it is not. 
%% 
%% We apply the adaptive LOG encoding here such that. This encoding translates
%% naturally to Bule. We introduce a generator fact {\verb bit(D,I,P)} that is the 
%% bit representation of the  number $D$ with the $I$s bit to 1 (resp. 0). 
%% 
%% \lstinputlisting{programs/prime.bul}
%% 
%% \subsection{Synthesis: Unit Propagation Complete}
%% 
%% Boolean function synthesis is a common application of QBF.
%% For SAT programming one of the quality aspect of the CNF decomposition is the strenght of unit propagation on any partial assignment. 
%% In this section we show how to encode whether a concrete CNF formula is unit propagation complete (UPC), i.e. UP detects inconsistencies on any partial assignment (citation needed).  
%% For this we program a kind of meta reasoning, which demonstrates the power of QBF. 
%% 
%% \lstinputlisting{programs/gac.bul}
%% 
%% \subsection{Approx Model Counting}
%% 
%% Find a cool probability problem where we define two SAT problems. 
%% One for the nominator, one for the denominator. 
%% 
%% \section{Complexity}
%% 
%% \abd{Analyse} and \jean{check}. 
%% 
%% In Table \ref{tab:complexity} we give an overview of the complexity classes
%% that each fragment of is complete for. The proof follows from the reduction of
%% Turing machines into an expression in Bule using subset of constructs.
%% 
%% \begin{table}
%%   \caption{Complexity Results}
%%   \centering
%%   \label{tab:complexity}
%%   \begin{tabular}{lccc}
%%     \toprule
%%     Clauses  & \multicolumn{1}{c}{Horn}  & \multicolumn{2}{c}{Any} \\
%%     \cmidrule{3-4}
%%     $\forall$ Quantifiers   & $\cmark$ & $\xmark$ & $\cmark$ \\
%%     \midrule
%%     \bflat   & \ptime    & \np       & \pspace  \\
%%     \bcore   & \exptime & \nexptime  & \expspace  \\
%%     \bfull   & &     &   \\
%%     \bottomrule
%%   \end{tabular}
%% \end{table}
%% 
%% \lstinputlisting{programs/turing/core.bul}
%% \lstinputlisting{programs/turing/busy-2-2.bul}
%% 
%% \section{Related Works and Citations and Ideas}
%% 
%% ASP: Bule smoothly avoids the difficulty induced by stable models, minimal models, well-founded 
%% semantics and loop formulas. From a logic programming point of view we treat tight, stratified, normal logic programs?
%% 
%% \begin{itemize}
%%     \item Writing Declarative Specification for Clauses: \cite{Gebser16} 
%%         \begin{itemize}
%%             \item The proposed language is a syntactic alternative to \bcore
%%             \item It uses translation forth and back between ASP
%%             \item no complexity for non-ground language analysed. 
%%             \item Does not distinguish syntactically between grounding facts and search variables as we do. 
%%             \item \bnice provides convenient modelling support that their language does not have. Implicit variables for instances. 
%%         \end{itemize}
%%     \item ASP vs SAT: Why are there so many Loop Formulas. (reachability to SAT without additional variables is large): \cite{Lifschitz04} .
%%     \item Programming a SAT formula via frameworks (e.g. \cite{Pysat18},
%%         more?), then this is a imperative description, but NOT a declarative
%%         one! Complexity results on Python are undecided of course. 
%%     \item Disjunctive Datalog, DLV \cite{Eiter97} or even just Datalog \cite{Gottlob89}
%%     \item Relate it to Complexity of different flavors of logic programming \cite{Gottlob01}
%%     \item Assat, translating ASP programs to SAT \cite{Lin04}
%%     \item \cite{Janhunen11}. Compact Translations of Non-disjunctive Answer Set
%%         Programs to Propositional Clauses
%%     \item FO(ID). First order logic with bounds (guards) \cite{Wittocx10}
%%     \item Predicate Logic as a modeling language. IDP System. \cite{Cat18}
%%     \item Answer Set Programming Lparse/Gringo \cite{Gebser15}, \cite{Ferraris05}
%% %    \item QBF Solvers \cite{Lonsing17,Tentrup15}
%%     \item Lazy Clause Generation Interleaving Grounding and Search \cite{Cat15}
%%     \item Relationship with Effectively Propositional Logic (EPL) \cite{Moura08}? 
%%     \item http://picat-lang.org/
%% \end{itemize}
%%  
%% \bibliographystyle{plain}
%% \bibliography{main}
%% 
%% \end{document}
