\documentclass[runningheads]{llncs}

\usepackage{misc}
\usepackage{cite}
\usepackage{amsmath,amssymb,amsfonts}
\usepackage{algorithmic}
\usepackage{graphicx}
\usepackage{textcomp}
\usepackage{xcolor}
\usepackage{xspace}
\usepackage{color}
\usepackage{multirow}
\usepackage{textcomp}
\usepackage{booktabs}
\newcommand{\nn}{\raisebox{0.5ex}{\texttildelow}}
\definecolor{codegreen}{rgb}{0,0.6,0}
\definecolor{codegray}{rgb}{0.5,0.5,0.5}
\definecolor{codepurple}{rgb}{0.58,0,0.82}
\definecolor{backcolour}{rgb}{0.95,0.95,0.92}
\usepackage{listings}
\lstdefinestyle{mystyle}{
    backgroundcolor=\color{backcolour},   
    commentstyle=\color{codegreen},
    keywordstyle=\color{magenta},
    numberstyle=\tiny\color{codegray},
    stringstyle=\color{codepurple},
    basicstyle=\ttfamily\footnotesize,
    breakatwhitespace=false,         
    breaklines=true,                 
    captionpos=b,                    
    keepspaces=true,                 
    numbers=left,                    
    numbersep=5pt,                  
    showspaces=false,                
    showstringspaces=false,
    showtabs=false,                  
    tabsize=2,
    literate={~}{\nn}{1}
    %literate={~}{\raise.17ex\hbox{$\scriptstyle\sim$}}{1},
    %literate={~} {$\sim$}{1}
%    literate={~}{\textasciitilde}{1},
%    literate={~}{$\neg$}{1},
}
\lstset{style=mystyle}

\newcommand\bigforall{\mbox{\Large $\mathsurround0pt\forall$}}
\newcommand\bigexists{\mbox{\Large $\mathsurround0pt\exists$}}

\newtheorem{example}{Example}

\newcommand{\complexityclass}[1]{\textsf{\footnotesize #1}\xspace}
\newcommand{\ptime}{\complexityclass{P}}
\newcommand{\np}{\complexityclass{NP}}
\newcommand{\pspace}{\complexityclass{PSPACE}}
%\newcommand{\exptime}{\complexityclass{EXPTIME}}
%\newcommand{\nexptime}{\complexityclass{NEXPTIME}}
%\newcommand{\nexpspace}{\complexityclass{EXPSPACE}}



\title{ QBF Programming with \emph{Bule}}

\author{Jean Christoph Jung\inst{1} \and Valentin Mayer-Eichberger\inst{2} \and
Abdallah Saffidine\inst{3}}
\authorrunning{Jung et.al.}

\institute{Universit\"at Bremen, Germany \and Technische Universit\"at Berlin, Germany \and University of New South Wales, Sydney, Australia }

\begin{document}

\maketitle

\begin{abstract}
    Bule introduces the concept of QBF Programming - a paradigm that complements QBF solving with an intuitive modelling language. 
    The language is easy to learn and has a transparent grounding process that unrolls firt order term variables to a ground CNF formula with a quantifier prefix. 
%    Bule separates domain predicates for the grounding process and search variables syntactically. 
    The semantics of the language is strict and well-behaved and we complexity classes for both ground and non-ground programs. 
%    Our implementation can connect to any QBF solver and in case of only existential variables SAT solver. 
    Finally, Bule is a tool to learn QBF and SAT Programming due to its easy format and flexible integration with any QBF and SAT solvers. 
\end{abstract}

%% Motivation: 
%% 
%% simpler and more strict, yet expressive enough. So it's more well-behaved and defined. 
%% This could lead solver developers to go for the actual non-ground specification of the problem. 
%% Orthogonal to this hierarchy is the hierarchy of quantifier levels. 
%% And a problem can be horn (max 1 positive literal per clause) or not. 
%% 
%% focus on grounding language (not fast grounder yet)
%% 
%% simpler and more strict: 
%% * non-recursive 
%% * separation of grounding and search variables
%%     * Because we make this distinction, we can rule out recursion. 
%% * no functional terms
%%  
%% 
%% Contributions
%% - SAT Programming
%% actually, a QSAT Programming Language: PSpace 
%% We fix the hebrand terms via (LOOK AT THIS: some problems cant be expressed nicely in BULE?)
%% There is a working implementation, used in practice. 
%% Formal analysis of non-ground modelling language 
%% Much cleaner than ASP for NON-Ground modelling lanugage 
%% ASP(ground) is sigma2P, 
%% ASP is not intended for modelling PSPACE languages
%% Loop formulas> https://dl.acm.org/doi/abs/10.1145/1131313.1131316 SAT vs ASP
%% 1. Fix Format 
%% 2. fix the Examples for a MVP(minimal viable paper).
%% Why iterators in nice and not core? (can be expressed with extra variables)
%% Debate: Should we allow the following: 
%% Exponents? facts[2..2^K].
%% Free variables?
%% Syntax 

\section{Introduction and Motivation}

%\paragraph{Introduction}

Creating correct formulas for SAT and QBF solvers is difficult not just for beginners. 
The advantages to the low level input format DIMACS, such as uniformity and simple semantics, is also its drawback.  
DIMACS format is hard to read and debug, and the original meaning of the formula is lost. 
To the best of our knowlede, non-ground model languages for SAT are rare and in the case of QBF they do not exist. 
There are frameworks that wrap SAT solvers and help with programming encodings such as PySAT \cite{PySAT2018} but they do not provide a formal modelling language. 
The modelling and solving framework \emph{Answer Set Programming} ASP provides non-ground modelling languages such as LParse \cite{Lparse} or Gringo \cite{Gebser15}. 
However, the semantics of non-monotonic logics is complicated and not needed for many NP-Complete problems. 
 
We propose a modelling language for SAT and QBF community with simpler syntax and semnatics than ASP. 
In this extended abstract, we introduce the syntax of the language and provide examples. We also discuss syntactical variants of the language that lead to different complexity classes. 

\paragraph{Running Example}

We start with an example: 
Given a graph compute all connections. 
Bule follows the fashion of logic programming to write term variable starting capital letters, and predicates and terms starting with lower case letters. 
Reachability Code Cybic. 

\section{The Bule Language}


\begin{align*}
    & \bigwedge_{\text{Generator}} \bigvee_{\text{Iterator}} \text{Literal} \\ 
\end{align*}

Brief Syntax and Semantics
EBNF ? 

\subsection{Complexity}

\paragraph{Table}

\abd{Analyse} and \jean{check}. 

In Table \ref{tab:complexity} we give an overview of the complexity classes
that each fragment of is complete for. The proof follows from the reduction of
Turing machines into an expression in Bule using subset of constructs.

\begin{table}
  \caption{Complexity Results}
  \centering
  \label{tab:complexity}
  \begin{tabular}{lccc}
    \toprule
    Clauses  & \multicolumn{1}{c}{Horn}  & \multicolumn{2}{c}{Any} \\
    \cmidrule{3-4}
    $\forall$ Quantifiers   & $\cmark$ & $\xmark$ & $\cmark$ \\
    \midrule
    \bflat   & \ptime    & \np       & \pspace  \\
    \bcore   & \exptime & \nexptime  & \expspace  \\
    \bfull   & &     &   \\
    \bottomrule
  \end{tabular}
\end{table}


\section{Examples}

\paragraph{Counter}
\paragraph{SAT Reachability}
\paragraph{QBF Reachability}
\paragraph{Turing Machine}

\section{Related Work}
 \begin{itemize}
     \item Writing Declarative Specification for Clauses: \cite{Gebser16} 
         \begin{itemize}
             \item The proposed language is a syntactic alternative to \bcore
             \item It uses translation forth and back between ASP
             \item no complexity for non-ground language analysed. 
             \item Does not distinguish syntactically between grounding facts and search variables as we do. 
             \item \bnice provides convenient modelling support that their language does not have. Implicit variables for instances. 
         \end{itemize}
     \item ASP vs SAT: Why are there so many Loop Formulas. (reachability to SAT without additional variables is large): \cite{Lifschitz04} .
     \item Programming a SAT formula via frameworks (e.g. \cite{PySAT18},
         more?), then this is a imperative description, but NOT a declarative
         one! Complexity results on Python are undecided of course. 
     \item Disjunctive Datalog, DLV \cite{Eiter97} or even just Datalog \cite{Gottlob89}
     \item Relate it to Complexity of different flavors of logic programming \cite{Gottlob01}
     \item Assat, translating ASP programs to SAT \cite{Lin04}
     \item \cite{Janhunen11}. Compact Translations of Non-disjunctive Answer Set
         Programs to Propositional Clauses
     \item FO(ID). First order logic with bounds (guards) \cite{Wittocx10}
     \item Predicate Logic as a modeling language. IDP System. \cite{Cat18}
     \item Answer Set Programming Lparse/Gringo \cite{Gebser15}, \cite{Ferraris05}
 %    \item QBF Solvers \cite{Lonsing17,Tentrup15}
     \item Lazy Clause Generation Interleaving Grounding and Search \cite{Cat15}
     \item Relationship with Effectively Propositional Logic (EPL) \cite{Moura08}? 
     \item http://picat-lang.org/
 \end{itemize}

\section{Conclusion and Further Work}

\bibliographystyle{plain}
\bibliography{main}

\end{document}

